% "Станет проще"

\documentclass[a4paper,12pt]{article} % тип документа

% report, book

%  Русский язык

\usepackage[T2A]{fontenc}			% кодировка
\usepackage[utf8]{inputenc}			% кодировка исходного текста
\usepackage[english,russian]{babel}	% локализация и переносы


% Математика
\usepackage{amsmath,amsfonts,amssymb,amsthm,mathtools} 


\usepackage{wasysym}

%Заговолок
\author{Дерзай знать}
\title{Общие принципы и математика в \LaTeX{}}
\date{\today}


\begin{document} % начало документа

\maketitle
\newpage

Наша первая строчка.\\[2cm]
Вторая \hspace{20pt} строчка.

Важное можно выделить \textbf{жирным}

Эстеты могут воспользоваться \textit{курсивом}

Прагматичные могут \underline{подчеркнуть}

Можно даже \fbox{в рамочку}

Дефис -- не тире.

Кавычки это не Shift+2. Кавычки это <<так>>

\section{Мир формул}

Наша первая формула $100+100=200$, ага.

\[ 100+100=200 \]

\begin{equation}\label{eq: ifagor}
a^2+b^2=c^2
\end{equation}

Теорему Пифагора \eqref{pifagor} вы знаете с 8 класса\footnote{Определенно знали}. Эта теорема упоминается на странице \pageref{pifagor}.

\subsection{Дроби}

$\frac{1}{3}+\frac{1}{3}=\frac{2}{3}$. Вот вам и дроби\footnote{А это с пятого класса.}. {\scriptsize Так некрасиво.} {\Large Красиво так}:

\[ \frac{1}{3}+\frac{1}{3}=\frac{2}{3} \]


\subsection{Скобки}

\[ (2+3)\cdot 5=25 \]

\[ \left[\frac{4}{2}+3\right]\cdot 5=25 \]

\[ \{2+3\}\cdot 5=25 \]

\subsection{Индексы}

\[ m_1, m_{12}, c^2, c^{22} \]

\subsection{Стандартные функции}

\[ \sin x=0 \]
\[ \arctg x=\sqrt[5]{3} \]
\[ \log_{x-1}{(x^2-3x-4)}\geqslant 2 \]
\[ \lg 10=\ln e \]

\subsection{Функции покрупнее}

$\sum_{i=1}^{n}a_i+b_i$

\[ \sum_{i=1}^{n}a_i+b_i \]

$I=\int r^2dm$

\[I=\int r^2dm \]

\[I=\int_{0}^{1} r^2dm \]

\[I=\int\limits_{0}^{1} r^2dm \]

\subsection{Символы}

\[2\times 2\neq 5 \]

\[x \cap y,  x \cup y\]

\[x\in (-\infty; 0)\]

\[ \triangle ABC = \triangle A_1B_1C_1 \Rightarrow \angle A= \angle A_1\]

\smiley

\end{document} % конец документа
